\documentclass[10pt,fleqn]{book} % Default font size and left-justified equations
\usepackage[%
    pdftitle={Cycle 01 : Introduction aux grandeurs physiques},
    pdfauthor={Xavier Pessoles}]{hyperref}

\input{style/new_style}
\input{style/macros_SII}

\fichetrue
%\fichefalse

%\proftrue
\proffalse

\tdtrue
%\tdfalse

%\courstrue
\coursfalse

\newcommand{\bfsf}[1]{\textbf{\textsl{#1}}}%{\textbf{\textsf{#1}}}

% -------------------------------------
% Déclaration des titres
% -------------------------------------

\def\discipline{Sciences \\ Industrielles de \\
l'Ingénieur}
\def\xxtete{Sciences Industrielles de l'Ingénieur}
\def\classe{PTSI}
\def\xxnumpartie{Cycle 01}
\def\xxpartie{Découverte des systèmes pluritechnologiques -- Introduction aux grandeurs physiques}

\def\xxnumchapitre{Chapitre 1 \& 2}
\def\xxchapitre{\hspace{.12cm} Grandeurs mécaniques et électriques}

\def\xxposongletx{2}
\def\xxposonglettext{1.45}
\def\xxposonglety{13}%10

\def\xxonglet{Cycle 1 -- Ch. 1 \& 2}

\def\xxactivite{Colle 4}
\def\xxauteur{\textsl{Xavier Pessoles}}

\def\xxcompetences{%
\textsl{%
\textbf{Savoirs et compétences :}
\begin{itemize}[label=\ding{112},font=\color{ocre}] 
\item -- %Alg -- C15 : Récursivité : avantages et inconvénients.
\end{itemize}
}}

\def\xxfigures{
}%figues de la page de garde

\def\xxpied{%
Cycle 1 -- Introduction aux grandeurs physiques \\
Ch. 1 \& 2 : Grandeurs mécaniques et électriques -- \xxactivite%
}

\setcounter{secnumdepth}{5}
\def\xxtitreexo{Exercices d'application}
\def\xxsourceexo{}
%---------------------------------------------------------------------------
\begin{document}
\input{style/new_pagegarde}
\vspace{7cm}
\pagestyle{fancy}
\thispagestyle{plain}


\def\columnseprulecolor{\color{ocre}}
\setlength{\columnseprule}{0.4pt} 
\begin{multicols}{2}
%---------------------------------------------------------------------------

\section*{Exercice 1 -- Mouvement de translation}

Joe Dupont conduit une voiture à $50\; \text{km}\,\text{h}^{-1}$ dans une rue horizontale. La voiture a une masse de $1\,060\; \text{kg}$. Soudain, il freine pour s’arrêter.  On suppose que la décélération est constante pendant tout le freinage ($a = -2\; \text{m}\,\text{s}^{-2}$).

\subparagraph{}
\textit{Indiquer la direction et le sens de la force exercée sur la voiture, calculer son intensité.}

\subparagraph{}
\textit{Calculer la durée du freinage.}

\subparagraph{}
\textit{Calculer la distance du freinage.}


\section*{Exercice 2 -- Calcul de moments}

\setcounter{subparagraph}{0}
On donne la structure suivante : 
\begin{center}
\includegraphics[width=.6\linewidth]{images/moment8}
\end{center}

\subparagraph{}
\textit{Déterminer $\vect{\mathcal{M}\left(A,\vect{R} \right)}$.}

On donne la structure suivante : 
\begin{center}
\includegraphics[width=.6\linewidth]{images/fig_03}
\end{center}


\subparagraph{}
\textit{Déterminer $\vect{\mathcal{M}\left(A,\vect{F} \right)}$ puis $\vect{\mathcal{M}\left(O,\vect{F} \right)}$.}




\section*{Exercice 3 -- Circuit électrique de voiture}
\setcounter{subparagraph}{0}
Le schéma ci-dessous est le schéma partiel d'un circuit électrique de voiture :
\begin{center}
\includegraphics[width=.8\linewidth]{images/fig_04}
\end{center}

On donne : $E_{\text{bat}}  = 13,8 \; \text{V}$, $R_{\text{bat}} = 20 \; \text{m}\Omega$, $R_{\text{all}}  = 2 \; \Omega$ et $R_{\text{fils}}  = 0,1 \; \Omega$.


\subparagraph{}
\textit{Déterminer la résistance équivalente des charges (allumage, phares, fils et dégivreur).}
\ifprof
\begin{corrige}
On a alors : 
\begin{center}
\includegraphics[width=.9\linewidth]{images/cor_01}
\end{center}
$\dfrac{1}{R_\text{eq}} =\dfrac{1}{R_\text{all}} +\dfrac{1}{R_\text{pha}} +\dfrac{1}{R_\text{fils}+R_\text{deg} }$, $\dfrac{1}{R_\text{eq}}=\dfrac{1}{2}+\dfrac{1}{0,6}+\dfrac{1}{0,4+0,1}=4,166$.
On a donc $R_\text{eq}=0,240 \;\Omega$.

\end{corrige}
\else
\fi


\subparagraph{}
\textit{Déterminer le courant fourni par le générateur..}

\subparagraph{}
\textit{Déterminer la tension aux bornes de chacune des résistances.}


\end{multicols}
%\begin{thebibliography}{2}
%\bibitem{1}{Patrick Beynet, \textit{Supports de cours de TSI 2}, Lycée Rouvière, Toulon.}
%\bibitem{2}{<< Mandel zool 08 satellite antenna >>. Sous licence CC BY-SA via Wikimedia Commons - \url{https://fr.wikipedia.org/wiki/Ensemble_de_Mandelbrot#/media/File:Mandel_zoom_08_satellite_antenna.jpg}}
%\bibitem{3}{\url{http://www.obside.fr/fractales/pages/Recursif/}}
%\end{thebibliography}
\end{document}

