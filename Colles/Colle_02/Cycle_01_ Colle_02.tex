\documentclass[10pt,fleqn]{book} % Default font size and left-justified equations
\usepackage[%
    pdftitle={Cycle 01 : Introduction aux grandeurs physiques},
    pdfauthor={Xavier Pessoles}]{hyperref}

\input{style/new_style}
\input{style/macros_SII}

\fichetrue
%\fichefalse

\proftrue
%\proffalse

\tdtrue
%\tdfalse

%\courstrue
\coursfalse

\newcommand{\bfsf}[1]{\textbf{\textsl{#1}}}%{\textbf{\textsf{#1}}}

% -------------------------------------
% Déclaration des titres
% -------------------------------------

\def\discipline{Sciences \\ Industrielles de \\
l'Ingénieur}
\def\xxtete{Sciences Industrielles de l'Ingénieur}
\def\classe{PTSI}
\def\xxnumpartie{Cycle 01}
\def\xxpartie{Découverte des systèmes pluritechnologiques -- Introduction aux grandeurs physiques}

\def\xxnumchapitre{Chapitre 1 \& 2}
\def\xxchapitre{\hspace{.12cm} Grandeurs mécaniques et électriques}

\def\xxposongletx{2}
\def\xxposonglettext{1.45}
\def\xxposonglety{13}%10

\def\xxonglet{Cycle 1 -- Ch. 1 \& 2}

\def\xxactivite{Colle 2}
\def\xxauteur{\textsl{Xavier Pessoles}}

\def\xxcompetences{%
\textsl{%
\textbf{Savoirs et compétences :}
\begin{itemize}[label=\ding{112},font=\color{ocre}] 
\item -- %Alg -- C15 : Récursivité : avantages et inconvénients.
\end{itemize}
}}

\def\xxfigures{
}%figues de la page de garde

\def\xxpied{%
Cycle 1 -- Introduction aux grandeurs physiques \\
Ch. 1 \& 2 : Grandeurs mécaniques et électriques -- \xxactivite%
}

\setcounter{secnumdepth}{5}
\def\xxtitreexo{Exercices d'application}
\def\xxsourceexo{}
%---------------------------------------------------------------------------
\begin{document}
\input{style/new_pagegarde}
\vspace{8cm}
\pagestyle{fancy}
\thispagestyle{plain}


\def\columnseprulecolor{\color{ocre}}
\setlength{\columnseprule}{0.4pt} 
\begin{multicols}{2}
%---------------------------------------------------------------------------

\section*{Exercice 1 -- Théorème de superposition}
On donne le schéma électrique suivant :
\begin{center}
\includegraphics[width=\linewidth]{images/fig_01}
\end{center}


\subparagraph{}
\textit{Donner la tension aux bornes de chacun des dipôles ainsi que les courants traversant chaque dipôle en fonction de $R_i$ et de $E_i$.}


\section*{Exercice 2 -- Calcul de moments}

\setcounter{subparagraph}{0}
On donne la structure suivante : 
\begin{center}
\includegraphics[width=.8\linewidth]{images/fig_02}
\end{center}

\subparagraph{}
\textit{Déterminer $\vect{\mathcal{M}\left(A,\vect{F} \right)}$.}



On donne la structure suivante : 
\begin{center}
\includegraphics[width=.8\linewidth]{images/fig_03}
\end{center}


\subparagraph{}
\textit{Déterminer $\vect{\mathcal{M}\left(B,\vect{F} \right)}$.}




\section*{Exercice 3 -- Solide en rotation}
\setcounter{subparagraph}{0}

 Une machine est entraînée par un moteur électrique de fréquence nominale $1\,500\; \text{tr}\,\text{min}^{-1}$. Celui-ci exerce au démarrage un couple moteur constant de $40\; \text{N}\,\,\text{m}$. Le moment d’inertie de l’ensemble de la chaîne cinématique rapporté à l’axe du rotor est de $12,5\; \text{kg}\,\text{m}^2$. Le couple résistant dû aux frottements est supposé constant et égal à $4\, \text{N} \, \text{m}$.

\subparagraph{}
\textit{Calculer l’accélération du moteur pendant le démarrage.}

\subparagraph{}
\textit{Calculer le temps mis pour atteindre la fréquence nominale.}


\end{multicols}
%\begin{thebibliography}{2}
%\bibitem{1}{Patrick Beynet, \textit{Supports de cours de TSI 2}, Lycée Rouvière, Toulon.}
%\bibitem{2}{<< Mandel zool 08 satellite antenna >>. Sous licence CC BY-SA via Wikimedia Commons - \url{https://fr.wikipedia.org/wiki/Ensemble_de_Mandelbrot#/media/File:Mandel_zoom_08_satellite_antenna.jpg}}
%\bibitem{3}{\url{http://www.obside.fr/fractales/pages/Recursif/}}
%\end{thebibliography}
\end{document}

