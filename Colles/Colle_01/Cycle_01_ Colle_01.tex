\documentclass[10pt,fleqn]{book} % Default font size and left-justified equations
\usepackage[%
    pdftitle={Cycle 01 : Introduction aux grandeurs physiques},
    pdfauthor={Xavier Pessoles}]{hyperref}

\input{style/new_style}
\input{style/macros_SII}

\fichetrue
%\fichefalse

\proftrue
%\proffalse

\tdtrue
%\tdfalse

%\courstrue
\coursfalse

\newcommand{\bfsf}[1]{\textbf{\textsl{#1}}}%{\textbf{\textsf{#1}}}

% -------------------------------------
% Déclaration des titres
% -------------------------------------

\def\discipline{Sciences \\ Industrielles de \\
l'Ingénieur}
\def\xxtete{Sciences Industrielles de l'Ingénieur}
\def\classe{PTSI}
\def\xxnumpartie{Cycle 01}
\def\xxpartie{Découverte des systèmes pluritechnologiques -- Introduction aux grandeurs physiques}

\def\xxnumchapitre{Chapitre 1 \& 2}
\def\xxchapitre{\hspace{.12cm} Grandeurs mécaniques et électriques}

\def\xxposongletx{2}
\def\xxposonglettext{1.45}
\def\xxposonglety{13}%10

\def\xxonglet{Cycle 1 -- Ch. 1 \& 2}

\def\xxactivite{Colle 1}
\def\xxauteur{\textsl{Xavier Pessoles}}

\def\xxcompetences{%
\textsl{%
\textbf{Savoirs et compétences :}
\begin{itemize}[label=\ding{112},font=\color{ocre}] 
\item -- %Alg -- C15 : Récursivité : avantages et inconvénients.
\end{itemize}
}}

\def\xxfigures{
}%figues de la page de garde

\def\xxpied{%
Cycle 1 -- Introduction aux grandeurs physiques \\
Ch. 1 \& 2 : Grandeurs mécaniques et électriques -- \xxactivite%
}

\setcounter{secnumdepth}{5}
\def\xxtitreexo{Exercices d'application}
\def\xxsourceexo{}
%---------------------------------------------------------------------------
\begin{document}
\input{style/new_pagegarde}
\vspace{8cm}
\pagestyle{fancy}
\thispagestyle{plain}


\def\columnseprulecolor{\color{ocre}}
\setlength{\columnseprule}{0.4pt} 
\begin{multicols}{2}
%---------------------------------------------------------------------------

\section*{Exercice 1}

Pour aller rechercher des produits dans leurs rayons, Amazon utilise des axes linéaires afin de déplacer un préhenseur.
\begin{center}
\includegraphics[width=\linewidth]{images/fig_01}
\end{center}

Les performances dynamique de l'axe demandées sont les suivantes : 
\begin{itemize}
\item vitesse linéaire maximale : $50 \; \text{m}\,\text{min}^{-1}$;
\item accélération linéaire maximale : $9,8 \; \text{m}\, \text{s}^{-2}$.
\end{itemize}

\begin{obj}
L'objectif de ce travail est de déterminer les caractéristiques du moteur (vitesse et couple) permettant d'atteindre ces performances.
\end{obj}

\subparagraph{}
\textit{Quelle est la vitesse maximale que l'axe peut atteindre en  $\text{m}\, \text{s}^{-1}$.}
\ifprof
\begin{corrige}
$V = 0,83 \, \text{ms}^{-1}$
\end{corrige}
\else
\fi

\subparagraph{}
\textit{Combien de temps l'axe met-il pour atteindre la vitesse maximale ?}
\ifprof
\begin{corrige}
$T_a =0,83/9,8 = 0,08 s$
\end{corrige}
\else
\fi

\subparagraph{}
\textit{Quelle distance l'axe parcourt-il pour atteindre la vitesse maximale ?}
\ifprof
\begin{corrige}
\end{corrige}
\else
\fi


\subparagraph{}
\textit{Quelle est la longueur minimale à commander pour que l'axe puisse atteindre la vitesse maximale ?}
\ifprof
\begin{corrige}
\end{corrige}
\else
\fi

\subparagraph{}
\textit{Proposer une longueur minimale de l'axe pour pouvoir profiter de ses performances dynamiques.}
\ifprof
\begin{corrige}
\end{corrige}
\else
\fi


\subparagraph{}
\textit{Tracer le profil de la position, de la vitesse et de l'accélération pour parcourir une distance de 50 cm. On cherchera à atteindre les performances maximales de l'axe. }
\ifprof
\begin{corrige}
\end{corrige}
\else
\fi


Un motoréducteur permet d'entraîner un système poulie -- courroie permettant de déplacer la charge. On considère :
\begin{itemize}
\item une charge de masse $1\; \text{kg}$;
\item un poulie de rayon $5\; \text{cm}$;
\item un réducteur de rapport de transmission $1:20$.
\end{itemize}

\begin{center}
\includegraphics[width=.9\linewidth]{images/fig_02}
\end{center}

\subparagraph{}
\textit{Déterminer le couple à fournir par la poulie pour déplacer la charge lorsque l'accélération est au maximum. }
\ifprof
\begin{corrige}
\end{corrige}
\else
\fi


\subparagraph{}
\textit{Déterminer la vitesse et le couple à fournir par le moteur en considérant que l'inertie du motoréducteur est négligeable. }
\ifprof
\begin{corrige}
\end{corrige}
\else
\fi


\subparagraph{}
\textit{Donner la méthode permettant de prendre en compte l'inertie $J$ du motoréducteur ? Quel serait l'impact de la prise en compte de cette hypothèse ? }
\ifprof
\begin{corrige}
\end{corrige}
\else
\fi

\section*{Exercice 2}

\setcounter{subparagraph}{0}
On donne le schéma électrique suivant :
\begin{center}
\includegraphics[width=.9\linewidth]{images/fig_03}
\end{center}

avec : 
\begin{multicols}{2}
\begin{itemize}
\item $R_1 = 1 \: \text{k}\Omega$;
\item $R_2 = 3 \: \text{k}\Omega$;
\item $R_3 = 4 \: \text{k}\Omega$;
\item $R_4 = 2 \: \text{k}\Omega$:
\item $E=10 \; \text{V}$.
\end{itemize}
\end{multicols}

\subparagraph{}
\textit{Déterminer la résistance équivalente.}

\subparagraph{}
\textit{Déterminer la tension aux bornes de chacune des résistances ainsi que le courant traversant chaque dipôle.}


\section*{Exercice 3}

\setcounter{subparagraph}{0}
On donne la structure suivante : 
\begin{center}
\includegraphics[width=.8\linewidth]{images/fig_04}
\end{center}


\subparagraph{}
\textit{Déterminer $\vect{\mathcal{M}\left(A,\vect{F} \right)}$.}



On donne la structure suivante : 
\begin{center}
\includegraphics[width=.8\linewidth]{images/fig_05}
\end{center}


\subparagraph{}
\textit{Déterminer $\vect{\mathcal{M}\left(O,\vect{F} \right)}$.}
\end{multicols}

%\begin{thebibliography}{2}
%\bibitem{1}{Patrick Beynet, \textit{Supports de cours de TSI 2}, Lycée Rouvière, Toulon.}
%\bibitem{2}{<< Mandel zool 08 satellite antenna >>. Sous licence CC BY-SA via Wikimedia Commons - \url{https://fr.wikipedia.org/wiki/Ensemble_de_Mandelbrot#/media/File:Mandel_zoom_08_satellite_antenna.jpg}}
%\bibitem{3}{\url{http://www.obside.fr/fractales/pages/Recursif/}}
%\end{thebibliography}
\end{document}

